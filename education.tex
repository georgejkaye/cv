\section*{Education}
\stroke

\begin{education}{%
    PhD in Computer Science, University of Birmingham
}{September 2019 - present}%
    \educationsubheader{Diagrammatic semantics for digital circuits}{%
        Supervisors:
        \href{https://www.cs.bham.ac.uk/~drg/}{Dan Ghica},
        \href{https://www.cs.bham.ac.uk/~backensm/}{Miriam Backens}%
    }%
    \noindent
    \todo[inline]{Make this spiel better}
    Working on defining a diagrammatic semantics for digital circuits using
    hypergraphs, based on the work of Ghica and Jung. The ultimate aim is to
    define an automatic rewriting system that can be used as an sound, complete,
    and importantly \textit{efficient}, operational semantics for digital
    circuits.
\end{education}

\vspace{0.4em}

\begin{education}{%
    MSci in Computer Science, University of Birmingham
}{September 2015 - July 2019}%
    \educationsubheader{%
        Honours, Class I%
    }{%
        Supervisor: \href{https://noamz.org}{Noam Zeilberger}%
    }%
    \todo[inline]{Make this more concise}
    \noindent
    \renewcommand{\arraystretch}{1.55}
    \begin{tabular}{rp{11.5cm}}
        \textbf{Dissertation (74\%)}
        &
        Created a set of tools to help investigation of terms from various
        fragments of the $\lambda$-calculus: a visualiser to display
        $\lambda$-terms as 3-valent maps, and a generator to display all the
        maps that satisfied certain criteria, such as number of crossings. URL:
        \textit{
            \href{https://georgejkaye.com/lambda-visualiser}{georgejkaye.com/lambda-visualiser}}
        \\
        \textbf{Compilers \& Languages (80\%)}
        &
        Covered the structure of a typical compiler, parsing, stack frames,
        typechecking and functional languages.
        \\
        \textbf{Functional Programming (84\%)}
        &
        Introduced \textsc{Haskell} and \textsc{OCaml} and concepts such as
        recursion, polymorphism and higher order functions.
        \\
        \textbf{Teaching CS in Schools (85\%)}
        &
        Eight days spent at local primary school
        {\href{https://www.oasisacademyhobmoor.org/}{Oasis Academy Hobmoor}},
        preparing and giving lessons to pupils in Years 3 and 5. The lessons
        introduced the pupils to \textsc{Scratch} and helped them to create
        their own games.
        \\
        \textbf{Reasoning (90\%)}
        &
        Introduction to automated theorem proving in proposition and predicate
        logic. As an assignment, implemented a tool to perform resolution proofs
        or DPLL model generation in \textsc{Java}.
        URL:
        \textit{\href{https://georgejkaye.com/theorem-prover}{georgejkaye.com/theorem-prover}}
        \\
        \textbf{Team Project (80\%)}
        &
        Worked in a team of six to develop a multiplayer game in \textsc{Java}.
        Designed and implemented the graphics and map creation engine, and
        integrated the components developed by the team together in the final
        stages of the project.
        URL: \textit{\href{https://georgejkaye.com/team-project}{georgejkaye.com/team-project}}                                           \\
        \textbf{Other modules}
        &
        Mathematical Techniques for Computer Science (78\%),
        Principles of Programming Languages (76\%),
        Models of Computation (78\%), Software Workshop (Java) (89\%),
        C/C++ (77\%), Machine Learning (83\%)
    \end{tabular}
\end{education}

\begin{education}{Malmesbury School}{September 2008 - May 2015}%
    \noindent\textbf{A-levels:}
    Mathematics A*, Further Mathematics A*, Physics A, Religious Studies (AS) A,
    Music (AS) B

    \noindent\textbf{GCSEs:}
    7A*s, 4As (including English and Maths)
\end{education}