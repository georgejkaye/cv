\newcounter{educations}
\newcommand{\educationtext}[3]{{\textbf{#1}}\\[0.15em]\textit{#2}\\[0.1em]\small{#3}}
\newcommand{\education}[4]{%
    \stepcounter{educations}
    \pgfmathtruncatemacro{\itemno}{\value{educations}}
    \node[
        below=\sectionsep of #4.south west,
        eventdottext]
    (item \itemno header)
    {\phantom{#1}};
    \node[
        below=\sectionsep of #4.south west,
        eventdottext
    ](item \itemno){\educationtext{#1}{#2}{#3}};
    \node[
        left=\timedotsep of item \itemno header,
        timedot
    ](time \itemno)
    {};
}
\newcounter{experiences}
\newcommand{\experiencetext}[4]{{\textbf{#1, #2}}\\[0.15em]\textit{#3}\\[0.1em]{\small#4}}
\newcommand{\experience}[5]{%
    \stepcounter{educations}
    \pgfmathtruncatemacro{\itemno}{\value{educations}}
    \node[
        below=\sectionsep of #5.south west,
        eventdottext]
    (item \itemno header)
    {\phantom{#1}};
    \node[
        below=\sectionsep of #5.south west,
        eventdottext
    ](item \itemno){\experiencetext{#1}{#2}{#3}{#4}};
    \node[
        left=\timedotsep of item \itemno header,
        timedot
    ](time \itemno)
    {};
}
\newcommand{\papertext}[4]{{\textbf{#1}}\\[0.15em]\textit{#2}, \url{#3}\\[0.1em]\small{#4}}
\newcommand{\paper}[5]{%
    \stepcounter{educations}
    \pgfmathtruncatemacro{\itemno}{\value{educations}}
    \node[
        below=\sectionsep of #5.south west,
        eventdottext]
    (item \itemno header)
    {\phantom{#1}};
    \node[
        below=\sectionsep of #5.south west,
        eventdottext
    ](item \itemno){\papertext{#1}{#2}{#3}{#4}};
    \node[
        left=\timedotsep of item \itemno header,
        timedot
    ](time \itemno)
    {};
}
\newcommand{\sectionstroke}[1]{%
    \begin{scope}[on background layer]
        \draw[line width=1.5pt,sidebar]
        let \p1=(#1.south west),
        \p2=(current page.east) in
        (\x1,\y1-\sectionheaderbelow) to (\x2,\y1-\sectionheaderbelow);
    \end{scope}
}
\begin{tikzpicture}[
        every node/.style={
                inner sep=0pt,
                outer sep=0pt
            },
        remember picture,
        overlay,
        shift={($(current page.north west)%
                    +(\sidewidth+2.75\margin+\timedotsep,-0.5\margin)$)}]
    \hypersetup{urlcolor=linkonlight}
    \node[sectionTitle] at (0,0) (title 1) {\cvsection{Education}};
    \node[left=\timedotsep of title 1,headerIcon] {\faGraduationCap};
    \sectionstroke{title 1}
    \education{%
        PhD in Computer Science, University of Birmingham%
    }{%
        September 2019 - July 2024
    }{%
        Using category theory to define a \emph{fully compositional framework}
        for digital circuits with delay and feedback, leading to a
        \emph{operational semantics} and \emph{equational theory} for circuits,
        as well as allowing for automatic reasoning about them with
        \emph{graph rewriting}.
    }{%
        title 1%
    }
    \education{%
        MSci in Computer Science, University of Birmingham (Honours, Class I)%
    }{%
        September 2015 - July 2019
    }{%
        Studied modules including
        Compilers \& Languages, Functional Programming, Software Workshop,
        C/C++, Operating Systems, Teaching CS In Schools
    }{%
        item 1%
    }
    \education{Malmesbury School}{September 2008 - May 2015}{%
        \textbf{A-levels:}
        Maths A*, Further Maths A*, Physics A, Religious Studies (AS) A,
        Music (AS) B
        \,
        \textbf{GCSEs:}
        7A*s, 4As (including English and Maths)
    }{%
        item 2%
    }
    \node[
        left=\timedotsep of item 3.south west,
        invisibletimedot
    ](time 4)
    {};
    \draw (time 1.center) to (time 4.center);

    \node[below=0.6cm of item 3.south west,sectionTitle] (title 2) {\cvsection{Experience}};
    \node[left=\timedotsep of title 2,headerIcon] {\faBriefcase};
    \node[below=0.6cm of item 3.south west,sectionTitle] (title 2 dummy) {\phantom{\cvsection{Education}}};
    \sectionstroke{title 2}
    \experience{Teaching Assistant}{University of Birmingham}{%
        September 2019 - June 2024
    }{%
        Supporting the
        \textit{Logical and Mathematical Foundations for Computer Science},
        \textit{Theories of Computation}, and
        \textit{Compilers \& Languages} modules.
        Have led regular group tutorials for 10-20 students and distributed
        the marking of assignments between the teaching assistants.
        Received awards in 2022 for going above and beyond my duties.
    }{title 2}
    \experience{Research Intern}{Huawei R\&D, Edinburgh}{%
        January 2021 - November 2023
    }{%
        Worked as part of the Programming Languages Lab in designing a hardware
        description language in Huawei's new programming language.
        Developed compiler optimisations for the new programming
        language using the large \textsc{C++} codebase.
    }{item 4}
    \node[
        left=\timedotsep of item 5.south west,
        invisibletimedot
    ](time 5)
    {};
    \draw (time 4.center) -- (time 5.center);

    \node[below=0.6cm of item 5.south west,sectionTitle] (title 3) {\cvsection{Papers}};
    \node[left=\timedotsep of title 3,headerIcon] {\faPencil*};
    \node[below=0.6cm of item 3.south west,sectionTitle] (title 3 dummy) {\phantom{\cvsection{Volunteering}}};
    \sectionstroke{title 3}

    \paper{A Fully Compositional Theory of Sequential Digital Circuits}{Arxiv preprint}{
        https://doi.org/10.48550/arXiv.2201.10456}{%
        We extend previous work modelling sequential circuits with delay and
        feedback as morphisms in a symmetric traced monoidal category by
        providing three sound and complete semantics: a denotational semantics
        in terms of stream functions, an operational semantics for processing
        the inputs to a circuit automatically, and an algebraic semantics for
        efficient reasoning.
    }{title 3}

    \paper{Rewriting Modulo Traced Comonoid Structure}{FSCD 2023}{
        https://doi.org/10.4230/LIPIcs.FSCD.2023.14}{%
        We adapt existing work on rewriting string diagrams using
        hypergraphs for settings with a traced comonoid structure.
        Examples of settings with such a structure include traced Cartesian
        categories, which are often used to model fixpoints; and the category
        of digital circuits.
    }{item 6}
    \node[
        left=\timedotsep of item 7.south west,
        invisibletimedot
    ](time 7)
    {};
    \draw (time 6.center) -- (time 7.center);

    \node[below=0.6cm of item 7.south west,sectionTitle] (title 4) {\cvsection{Volunteering}};
    \node[left=\timedotsep of title 4,headerIcon] {\faRocket};
    \node[below=0.6cm of item 3.south west,sectionTitle] (title 5 dummy) {\phantom{\cvsection{Volunteering}}};
    \sectionstroke{title 4}
    \education{Local Organiser, Twelfth Symposium on Compositional Structures}{%
        15-16 April 2024
    }{
        Raised funding to host the next edition of this conference at
        Birmingham.
        Responsibilities included maintaining the conference webpage,
        organising social events, booking and setting up rooms, arranging
        catering, and working with the committee to assemble the selection of
        talks.

        URL: \url{https://www.cl.cam.ac.uk/events/syco/12/}
    }{title 4}
    \education{Theory Seminar Organiser, University of Birmingham}{%
        March 2022 - March 2024
    }{
        Inviting speakers for the weekly research seminar, communicating with
        both speakers, hosts and audience, and setting up and maintaining the
        room during talks; also campaigned for extra funding to provide
        accommodation for speakers.
    }{item 8}
    \node[
        left=\timedotsep of item 9.south west,
        invisibletimedot
    ](time 9){};
    \draw (time 8.center) -- (time 9.center);

\end{tikzpicture}
