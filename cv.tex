\documentclass[10pt]{article}

\usepackage{ tgpagella }
\usepackage{ array, xcolor }
\usepackage[margin=1in]{ geometry }
\usepackage{ hyperref }

\newcommand{\stroke}{\hrule\vspace{1em}}

\title{\vspace{-0.75in}\Huge\textbf{GEORGE KAYE}}
\author{\href{mailto:georgejkaye@gmail.com}{\texttt{georgejkaye@gmail.com}} $\quad\bullet\quad$ \url{https://georgejkaye.com}}
\date{}
	
\definecolor{lightgray}{gray}{0.8}
\newcolumntype{L}{>{\raggedleft}p{0.2\textwidth}}
\newcolumntype{R}{p{0.8\textwidth}}
\newcommand\VRule{\color{lightgray}\vrule width 0.5pt}

\newenvironment{education}[3]
    {\noindent\large{\textbf{#1} \hfill \textit{#2 - #3}}
    \vspace{0.5em}
    }
    {\vspace{1em}}

\begin{document}
\maketitle

\section*{Profile}
\stroke

\section*{Education}
\stroke

\begin{education}{PhD in Computer Science, University of Birmingham}{September 2019}{present}

    \noindent\large\textit{Diagrammatic semantics for digital circuits} \hfill Supervisors: \href{https://www.cs.bham.ac.uk/~drg/}{Dan Ghica}, \href{https://www.cs.bham.ac.uk/~backensm/}{Miriam Backens}

    \vspace{0.5em}

    \noindent Working on defining a diagrammatic semantics for digital circuits using hypergraphs, based on the work of Ghica and Jung. The ultimate aim is to define an automatic rewriting system that can be used as an effective and efficient operational semantics for digital circuits.

    \vspace{0.5em}

    \noindent\textbf{Modules:} Research Skills (80\%)
\end{education}

\begin{education}{MSci in Computer Science, University of Birmingham}{September 2015}{July 2019}

\end{education}

\begin{education}{Malmesbury School}{September 2008}{May 2015}
\end{education}

\vspace{1em}

\section*{Work Experience}
\stroke


\end{document}