\documentclass[10pt]{article}

\usepackage{ tgpagella }
\usepackage{ array, xcolor }
\usepackage{ enumitem }
\usepackage[margin=0.7in]{ geometry }
\usepackage[colorlinks=true,urlcolor=blue]{ hyperref }

\newcommand{\stroke}{\hrule\vspace{1em}}
\newcommand{\tech}[3]{\lang{#1} & \textit{#2 proficiency:} #3}
\newcommand{\lang}[1]{\textbf{\textsc{#1}}}

\title{\vspace{-0.75in}\Huge\textbf{GEORGE KAYE}}
\author{(+44) 7544 646 100 $\hspace{0.3em}\bullet\hspace{0.3em}$ \href{mailto:georgejkaye@gmail.com}{georgejkaye@gmail.com} $\hspace{0.3em}\bullet\hspace{0.3em}$ \href{https://georgejkaye.com}{georgejkaye.com} $\hspace{0.3em}\bullet\hspace{0.3em}$ \textbf{GitHub:} \href{http://github.com/georgejkaye}{georgejkaye}}
\date{}
	
\definecolor{lightgray}{gray}{0.8}
\newcolumntype{L}{>{\raggedleft}p{0.2\textwidth}}
\newcolumntype{R}{p{0.8\textwidth}}
\newcommand\VRule{\color{lightgray}\vrule width 0.5pt}

\newenvironment{education}[3]
    {\noindent{\large{\textbf{#1} \hfill \textit{#2 - #3}}}
    \vspace{0.5em} 
    \noindent
    }
    {\vspace{0.5em}}

\newenvironment{work}[3]
{\noindent{\large{\textbf{#1} \hfill \textit{#2 - #3}}}
\vspace{0.5em}

}
{}

\begin{document}
\pagenumbering{gobble}

\maketitle

\section*{Profile}
\stroke

My clear logical thinking and adaptable approach to problem solving has enabled me to achieve a First in my Masters degree from the University of Birmingham.  I consistently manage my workload effectively and have the motivation to complete tasks on time. Moreover, my personable nature means I can work successfully in a team as well as independently. Currently, my research interests are in graphical calculi for compositional systems and the lambda calculus using monoidal categories, and reasoning diagrammatically with graph rewriting techniques.

\section*{Education}
\stroke

\begin{education}{PhD in Computer Science, University of Birmingham}{September 2019}{present}
    
    \noindent{\large\textit{Diagrammatic semantics for digital circuits} \hfill Supervisors: \href{https://www.cs.bham.ac.uk/~drg/}{Dan Ghica}, \href{https://www.cs.bham.ac.uk/~backensm/}{Miriam Backens}}

    \vspace{0.5em}

    \noindent Working on defining a diagrammatic semantics for digital circuits using hypergraphs, based on the work of Ghica and Jung. The ultimate aim is to define an automatic rewriting system that can be used as an sound, complete, and importantly \textit{efficient}, operational semantics for digital circuits.

    \vspace{0.5em}

    \noindent\textbf{Modules:} Research Skills (80\%)
\end{education}

\vspace{0.4em}

\begin{education}{MSci in Computer Science, University of Birmingham}{September 2015}{July 2019}

    \noindent{\large\textit{Honours, Class I} \hfill Supervisor:  \href{https://noamz.org}{Noam Zeilberger}}

    \vspace{0.5em}


    \renewcommand{\arraystretch}{1.55}  
    \begin{tabular}{rp{11.5cm}}
    \textbf{Dissertation (74\%)} & Created a set of tools to help investigation of terms from various fragments of the $\lambda$-calculus: a visualiser to display $\lambda$-terms as 3-valent maps, and a generator to display all the maps that satisfied certain criteria, such as number of crossings. URL: \textit{\href{https://georgejkaye.com/lambda-visualiser}{georgejkaye.com/lambda-visualiser}} \\
    \textbf{Compilers \& Languages (80\%)} & Covered the structure of a typical compiler, parsing, stack frames, typechecking and functional languages. \\
    \textbf{Functional Programming (84\%)} & Introduced \textsc{Haskell} and \textsc{OCaml} and concepts such as recursion, polymorphism and higher order functions. \\
    \textbf{Teaching CS in Schools (85\%)} & Eight days spent at local primary school {\href{https://www.oasisacademyhobmoor.org/}{Oasis Academy Hobmoor}}, preparing and giving lessons to pupils in Years 3 and 5. The lessons introduced the pupils to \textsc{Scratch} and helped them to create their own games. \\
    \textbf{Reasoning (90\%)} & Introduction to automated theorem proving in proposition and predicate logic. As an assignment, implemented a tool to perform resolution proofs or DPLL model generation in \textsc{Java}. URL: \textit{\href{https://georgejkaye.com/theorem-prover}{georgejkaye.com/theorem-prover}} \\
    \textbf{Team Project (80\%)} &  Worked in a team of six to develop a multiplayer game in \textsc{Java}. Designed and implemented the graphics and map creation engine, and integrated the components developed by the team together in the final stages of the project. URL: \textit{\href{https://georgejkaye.com/team-project}{georgejkaye.com/team-project}}  \\
    \textbf{Other modules} & Mathematical Techniques for Computer Science (78\%), Principles of Programming Languages (76\%), Models of Computation (78\%), Software Workshop (Java) (89\%), C/C++ (77\%), Machine Learning (83\%)
    \end{tabular}

\end{education}

\vspace{0.75em}

\begin{education}{Malmesbury School}{September 2008}{May 2015}
    \vspace{0.25em}

    \noindent\textbf{A-levels:} Mathematics A*, Further Mathematics A*, Physics A, Religious Studies (AS) A, Music (AS) B    

    \noindent\textbf{GCSEs:} 7A*s, 4As (including English and Maths)
\end{education}

\vspace{1em}

\newpage

\section*{Work Experience}
\stroke

\begin{work}{Teaching Assistant, University of Birmingham}{September 2019}{March 2019}
    \noindent Assisted in running the first year \textit{Mathematical Foundations for Computer Science} and final year \textit{Compilers \& Languages} \mbox{modules}. Duties involved organising and running regular group tutorials for 10-20 students, marking assignments and providing feedback.
\end{work}

\vspace{1em}

\begin{work}{Supermarket Assistant, Waitrose Malmesbury}{July 2014}{September 2019}
    \noindent Part-time work in the Replenishment team working with colleagues to efficiently manage and replenish stock. Role included helping customers find appropriate products, taking in and signing for deliveries, keeping the warehouse organised to facilitate stock management, cashier duties as and when required.    
\end{work}

\section*{Technical proficiencies}
\stroke


\renewcommand{\arraystretch}{1.5}
\begin{tabular}{rp{14.4cm}}
    \tech{Java}{High}{used extenively throughout my university course. Java was the primary language used for many large team projects and individual assignments. I have also used \lang{JUnit} for unit testing on larger projects.} \\
    \tech{JavaScript}{Intermediate}{Used to implement my dissertation using graph theory library \textsc{Cytoscape}, and for a circuit visualisation tool early in my PhD. Currently learning frameworks such as \lang{React}, \lang{Reason} and \lang{TypeScript} for both academic and personal projects.} \\
    \tech{OCaml/Haskell}{Intermediate}{Studied throughout my undergraduate degree and practised in various assignments throughout the modules.} \\
    \tech{Python}{High}{Used during university, and for small personal projects interacting with web APIs and Discord bots.} \\
    \tech{C/C++}{Intermediate}{Studied during university. Assignments completed include codng a basic HTTP server, implementing a Lisp-style expression tree parser and writing basic device drivers to interact with the kernel. I also use \lang{Valgrind} to ensure correct memory management in programs.} \\
    \tech{Git}{High}{Used for most personal and university projects.} \\
    \tech{VS Code}{High}{Preferred text editor, used for editing, debugging and building projects.} \\
    \tech{\LaTeX}{High}{Used regularly for writing notes and documents.}
\end{tabular}

\section*{Volunteering}
\stroke

\begin{work}{University society committees}{2016}{2019}
    \noindent Acted as Secretary of Middle-earth Society, organising large events such as inter-society quizzes and coordinating room bookings. Was also Vice-President of Doctor Who Society (2016-17), during which I organised a trip to the Doctor Who Experience in Cardiff and other socials to laser quest and bowling.
\end{work}

\vspace{1em}

\begin{work}{Maths breakfast club, Malmesbury School}{2013}{2015}
    \noindent Tutored Year 11 students preparing for their Maths GCSEs, providing support where needed and guiding them through solutions to questions they found challenging.
\end{work}

\vspace{-0.15em}

\section*{Interests}
\stroke

\begin{itemize}[noitemsep]
    \item Developing visualisers and tools for interesting systems and networks
    \item Playing piano: achieved a Merit at Grade 8 in July 2015
    \item Train journeys and long walks (especially along canals)
\end{itemize}

\end{document}