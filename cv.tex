\documentclass[10pt]{article}

\usepackage{ tgpagella }
\usepackage{ array, xcolor }
\usepackage{ enumitem }
\usepackage[margin=0.8in]{ geometry }
\usepackage[colorlinks=true,urlcolor=blue]{ hyperref }

\newcommand{\stroke}{\hrule\vspace{1em}}
\newcommand{\tech}[3]{\item[\textsc{#1}] \textit{#2 proficiency:} #3}
\newcommand{\lang}[1]{\textbf{\textsc{#1}}}

\title{\vspace{-0.75in}\Huge\textbf{GEORGE KAYE}}
\author{\href{mailto:georgejkaye@gmail.com}{georgejkaye@gmail.com} $\quad\bullet\quad$ \href{https://georgejkaye.com}{georgejkaye.com} $\quad\bullet\quad$ \textbf{GitHub:} \href{http://github.com/georgejkaye}{georgejkaye}}
\date{}
	
\definecolor{lightgray}{gray}{0.8}
\newcolumntype{L}{>{\raggedleft}p{0.2\textwidth}}
\newcolumntype{R}{p{0.8\textwidth}}
\newcommand\VRule{\color{lightgray}\vrule width 0.5pt}

\newenvironment{education}[3]
    {\noindent{\large{\textbf{#1} \hfill \textit{#2 - #3}}}
    \vspace{0.5em} 
    \noindent
    }
    {\vspace{1em}}

\newenvironment{work}[3]
{\noindent{\large{\textbf{#1} \hfill \textit{#2 - #3}}}
\vspace{0.5em}

}
{\vspace{1em}}

\begin{document}
\pagenumbering{gobble}

\maketitle

\section*{Profile}
\stroke

My clear logical thinking and adaptable approach to problem solving has enabled me to achieve a First in my Masters degree from the University of Birmingham.  I consistently manage my workload effectively and have the motivation to complete tasks on time. My personable nature means I can work successfully in a team as well as independently. My current research interests are in graphical calculi for compositional systems and the lambda calculus using monoidal categories, and reasoning diagrammatically with graph rewriting techniques.

\section*{Education}
\stroke

\begin{education}{PhD in Computer Science, University of Birmingham}{September 2019}{present}
    
    \noindent{\large\textit{Diagrammatic semantics for digital circuits} \hfill Supervisors: \href{https://www.cs.bham.ac.uk/~drg/}{Dan Ghica}, \href{https://www.cs.bham.ac.uk/~backensm/}{Miriam Backens}}

    \vspace{0.5em}

    \noindent Working on defining a diagrammatic semantics for digital circuits using hypergraphs, based on the work of Ghica and Jung. The ultimate aim is to define an automatic rewriting system that can be used as an effective and efficient operational semantics for digital circuits.

    \vspace{0.5em}

    \noindent\textbf{Modules:} Research Skills (80\%)
\end{education}

\begin{education}{MSci in Computer Science, University of Birmingham}{September 2015}{July 2019}

    \noindent{\large\textit{Honours, Class I}}

    \begin{description}
        \item[Compilers \& Languages (80\%)]\hfill
        
        Covered the structure of a typical compiler, parsing, stack frames, typechecking and functional languages.
        \item[Functional Programming (84\%), Elements of Functional Computing (98\%)]\hfill 

        Introduced \textsc{Haskell} and \textsc{OCaml} and concepts such as recursion, polymorphism and higher order functions.
        \item[Teaching Computer Science in Schools (85\%)] \hfill
        
        Eight days spent at local primary school {\href{https://www.oasisacademyhobmoor.org/}{Oasis Academy Hobmoor}}, preparing and giving lessons to pupils in Years 3 and 5. The lessons introduced the pupils to \textsc{Scratch} and helped them to create their own games.

        \item[Reasoning (90\%)] \hfill
         
        Introduction to automated theorem proving in proposition and predicate logic.

        \item[Models of Computation (78\%)]  \hfill
    
        Introduced several theoretical concepts such as the lambda calculus, state automata and Turing machines.
        \item[Team Project (80\%)] \hfill 
        
        Worked in a team of six to develop a multiplayer game in \textsc{Java}. I designed and implemented the graphics engine, and integrated the components developed by the team in teh final stages of the project. 

        \item[Other modules] \hfill

        Mathematical Techniques for Computer Science (78\%), Principles of Programming Languages (76\%), Software Workshop (Java) (89\%), C/C++ (77\%), Machine Learning (83\%)
        
    \end{description}

\end{education}

\begin{education}{Malmesbury School}{September 2008}{May 2015}
    \vspace{0.25em}

    \noindent\textbf{A-levels:} Mathematics A*, Further Mathematics A*, Physics A, Religious Studies (AS) A, Music (AS) B    

    \noindent\textbf{GCSEs:} 7A*s, 4As (including English and Maths)
\end{education}

\vspace{1em}

\newpage

\section*{Work Experience}
\stroke

\begin{work}{Teaching Assistant, University of Birmingham}{September 2019}{March 2019}
    \noindent Helping to run the first year \textit{Mathematical Foundations for Computer Science} and final year \textit{Compilers \& Languages} modules. Duties involved organising and running regular group tutorials for 10-20 students, marking assignments and providing feedback.
\end{work}

\begin{work}{Supermarket Assistant, Waitrose Malmesbury}{July 2014}{September 2019}
    \noindent Part-time work in the Replenishment team working with colleagues to efficiently manage and replenish stock. Role included helping customers find appropriate products, taking in and signing for deliveries,keeping the warehouse organised to facilitate stock management, cashier duties as and when required.    
\end{work}

\section*{Technical proficiencies}
\stroke

\begin{description}
    \tech{Java}{High}{used extenively throughout my university course. Java was the primart language used for many large team projects and individual assignments. I have also used \lang{JUnit} for unit testing on larger projects.}
    \tech{JavaScript}{High}{used to implement my Masters dissertation, a visualiser for linear lambda terms as rooted maps. I am also starting to learn frameworks such as \lang{React},\lang{Reason} and \lang{TypeScript} for academic and personal projects.}
    \tech{OCaml/Haskell}{Intermediate}{Studied throughout my undergraduate degree and practised in various assignments throughout the modules.}
    \tech{C/C++}{Intermediate}{Studied throughout . Assignments completed include coding a basic HTTP server, implementing a Lisp-style expression tree parser and writing basic device drivers to interact with the kernel. I also use \lang{Valgrind} to ensure correct memory management in programs.}
    \tech{Git}{High}{used for most personal and university projects (link provided above)}
    \tech{VS Code}{High}{My preferred text editor, used for editing, debugging and building projects.}
    \tech{\LaTeX}{High}{Used regularly for writing documents.}
\end{description}

\section*{Volunteering}
\stroke

\begin{work}{University society committees}{2016}{2019}
    \noindent Acted as Secretary of Middle-earth Society, organising large events such as inter-society quizzes and coordinating room bookings. Was also vice-president of Doctor Who Society (2016-17), during which I organised a trip to the Doctor Who Experience in Cardiff and other socials to laser quest and bowling.
\end{work}

\begin{work}{Maths breakfast club}{2013}{2015}
    \noindent Tutored Year 11 students preparing for their Maths GCSEs, providing support where needed and guiding them through solutions to questions they found challenging.
\end{work}

\section*{Interests}
\stroke

\begin{itemize}[noitemsep]
    \item Playing piano: achieved a Merit at Grade 8 in July 2015
    \item Walking and canals
    \item Bleh
\end{itemize}

\end{document}