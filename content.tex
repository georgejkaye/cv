
\newcounter{educations}
\newcommand{\test}{1}
\newcommand{\preveducation}{}
\newcommand{\education}[4]{
    \stepcounter{educations}
    \pgfmathtruncatemacro{\itemno}{\value{educations}}
    \node[
        below=\sectionsep of #4.south west,
        eventdottext]
        (item \itemno header)
        {\phantom{Evening}};
    \node[
        below=\sectionsep of #4.south west,
        eventdottext
    ](item \itemno){\cveducation{#1}{#2}{#3}};
    \node[
    left=\timedotsep of item \itemno header,
    timedot]
    (start)
    {};
}

\begin{tikzpicture}[
    every node/.style={
      inner sep=0pt,
      outer sep=0pt
    },
    remember picture,
    overlay,
    shift={($(current page.north west)%
             +(\sidewidth+2.75\margin+\timedotsep,-\margin)$)}]

    \node[sectionTitle] at (0,0) (title 1) {\cvsection{Education}};
    \node[left=\timedotsep of title 1,headerIcon] {\faGraduationCap};
    \begin{scope}[on background layer]
        \draw[line width=2pt,sidebar]
        let \p1=(title 1.south west),
            \p2=(current page.east) in
        (\x1,\y1-\sectionheaderbelow) to (\x2,\y1-\sectionheaderbelow);
    \end{scope}
    \education{%
        PhD in Computer Science, University of Birmingham%
    }{%
        September 2019 - present
    }{%
        Working on defining a diagrammatic semantics for digital circuits using
        hypergraphs, based on the work of Ghica and Jung. The ultimate aim is to
        define an automatic rewriting system that can be used as an sound, complete,
        and importantly \textit{efficient}, operational semantics for digital
        circuits.
        \todo[inline]{Spruce up}
    }{%
        title 1%
    }
    \education{%
        MSci in Computer Science, University of Birmingham%
    }{%
        September 2015 - July 2019
    }{%
        \textbf{Relevant modules:}
            Compilers \& Languages, Functional Programming, Software Workshop,
            C/C++, Operating Systems
    }{%
        item 1%
    }
    \education{Malmesbury School}{September 2008 - May 2015}{%
        \textbf{A-levels:}
        Maths A*, Further Maths A*, Physics A, Religious Studies (AS) A,
        Music (AS) B
        \,
        \textbf{GCSEs:}
        7A*s, 4As (inc. English and Maths)
    }{%
        item 2%
    }
\end{tikzpicture}